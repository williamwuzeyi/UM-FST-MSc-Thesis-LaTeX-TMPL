\chapter{\uppercase{[Type Chapter Title]}}
[Type Chapter]
\section{\uppercase{[Type Chapter Section Title]}}
[Type Chapter Section]
\subsection{\textsc{[Type Chapter Subsection Title]}}
[Type Chapter Subsection]
\subsubsection{[Type Chapter Subsubsection Title]}
[Type Chapter Subsubsection]
\subsubsection{Figure}
Figure \ref{fig:2:1}, Figure \ref{fig:2:2a} and Figure \ref{fig:2:2}(b)--(c) show the same example.

\begin{figure}[!htb]
	\centering
	\includegraphics[width=0.9\linewidth]{Figures/Example}
	\caption{Example of a figure.}
	\label{fig:2:1}
\end{figure}

\begin{figure}[!htb]
	\centering
	\subfloat[]{\includegraphics[width=0.45\linewidth]{Figures/Example}
		\label{fig:2:2a}}
	\hfil
	\subfloat[]{\includegraphics[width=0.45\linewidth]{Figures/Example}
		\label{fig:2:2b}}
	\hfil
	\subfloat[]{\includegraphics[width=0.45\linewidth]{Figures/Example}
		\label{fig:2:2c}}
	\hfil
	\subfloat[]{\includegraphics[width=0.45\linewidth]{Figures/Example}
		\label{fig:2:2d}}
	\caption{Example of multi-figure.}
	\label{fig:2:2}
\end{figure}